% !TEX root =  main.tex

\newcommand{\errbits}{\log (1/\epsilon)}
\newcommand{\opt}{n\errbits}
\renewcommand{\epsilon}{\varepsilon}

% \newcommand{\defn}[1]{\textit{#1}}
\newcommand{\defn}[1]{{\textit{\textbf{\boldmath #1}}}\xspace}
\newtheorem{theorem}{Theorem}
\newtheorem{lemma}[theorem]{Lemma}
\newtheorem{corollary}[theorem]{Corollary}
\newtheorem{claim}[theorem]{Claim}
\newtheorem{proposition}[theorem]{Proposition}
\newtheorem{assumption}[theorem]{Assumption}
\newtheorem{definition}[theorem]{Definition}
\newtheorem{invariant}[theorem]{Invariant}
\newtheorem{observation}[theorem]{Observation}

\makeatletter
\algrenewcommand\ALG@beginalgorithmic{\footnotesize}
\makeatother


\definecolor{magenta4}{rgb}{0.5625,0,0.5625}
\definecolor{green4}{rgb}{0,0.5625,0}
\definecolor{orange4}{rgb}{0.98,0.31,0.09}
\definecolor{powderblue}{rgb}{0.69,0.88,0.9}


\newcommand{\change}[2]{\sout{#1}\xspace\textcolor{blue}{#2}}
%\newcommand{\change}[2]{#2}

%\newenvironment{scaddition}[1]{\textcolor{blue}{#1}}
\newenvironment{scaddition}{}{}

\newenvironment{addition}{}{}

% see http://latexcolor.com to define more colors



%% comment the next few lines to remove all our comments

\newcommand{\sysname}{GPU-based quotient filter\xspace}
\newcommand{\Sysname}{GPU-based Quotient Filter\xspace}
\newcommand{\qf}{quotient filter\xspace}
\newcommand{\cqf}{counting quotient filter\xspace}
\newcommand{\bloom}{Bloom filter\xspace}


\newcommand{\poly}[1]{\textrm{poly}{(#1)}}
\newcommand{\polylog}[1]{\textrm{polylog}{(#1)}}

\newcommand{\para}[1]{\smallskip\noindent\textbf{#1.}}

